\documentclass[8pt, a4paper, twocolumn]{article}
\usepackage{amsmath,amsfonts,amsthm}
\usepackage{hyperref}
\usepackage{geometry}
\geometry{
	top=5mm,
	bottom=5mm,
	left=5mm,
	right=5mm,
	%includehead,
	%includefoot,
	%showframe,
}
\setlength{\columnsep}{5mm}

\usepackage{listings}
\usepackage{xcolor}

\definecolor{codegreen}{rgb}{0,0.6,0}
\definecolor{codegray}{rgb}{0.5,0.5,0.5}
\definecolor{codepurple}{rgb}{0.58,0,0.82}
\definecolor{backcolour}{rgb}{0.98,0.98,0.96}

\lstdefinestyle{mystyle}{
    backgroundcolor=\color{backcolour},   
    commentstyle=\color{codegreen},
    keywordstyle=\color{magenta},
    numberstyle=\tiny\color{codegray},
    stringstyle=\color{codepurple},
    basicstyle=\ttfamily\footnotesize,
    breakatwhitespace=false,         
    breaklines=true,                 
    captionpos=b,                    
    keepspaces=true,                 
    numbers=left,                    
    numbersep=5pt,                  
    showspaces=false,                
    showstringspaces=false,
    showtabs=false,                  
    tabsize=2
}

\lstset{style=mystyle}


\title{C++ Cheatsheet}
\author{Soumitra Das}
\date{\today}

\begin{document}
\maketitle
\section{Initial Template}
Uncomment line 5-9 if external library is needed.
\lstinputlisting[language=C++]{src/template.cpp}
\section{STL Library}

\subsection{Containers}
\paragraph{vector}
\paragraph{deque}
\paragraph{list}
\paragraph{forward\_list}
\paragraph{map}
\paragraph{unordered\_map}
\paragraph{multimap}
\paragraph{unordered\_multimap}
\paragraph{set}
\paragraph{unordered\_set}
\paragraph{multiset}
\paragraph{unordered\_multiset}
\paragraph{stack}
\paragraph{queue}
\paragraph{priority\_queue}
\paragraph{pair}
\paragraph{tuple}
\paragraph{tree}

\subsection{Algorithms}
\paragraph{sort}
\paragraph{reverse}
\paragraph{max\_element}
\paragraph{min\_element}
\paragraph{accumulate}
\paragraph{count}
\paragraph{find}
\paragraph{binary\_search}
\paragraph{lower\_bound}
\paragraph{upper\_bound}
\paragraph{next\_permutation}
\paragraph{prev\_permutation}
\paragraph{partition}
\paragraph{stable\_partition}
\paragraph{rotate}
\paragraph{min}
\paragraph{max}
\paragraph{swap}
\paragraph{\_\_gcd}
\paragraph{\_\_builtin\_popcount}

\section{Algorithms}
\subsection{Fibonacci numbers}
if $F_n$ is the $n$'th Fibonacci number, where $F_0=0$ and $F_1=1$, then
$$F_{n+k}=F_kF_{n+1}+F_{k-1}F_n$$
for any $n,k\in\mathbb{N}$.
\subsection{Geometric Transformation of points}
Point $(x,y,z)$ can be transformed by matrix multiplication 
\begin{equation*}
\begin{bmatrix}
x & y & z & 1
\end{bmatrix}\times\begin{bmatrix}
a_{11} & a_{12} & a_{13} & a_{14}\\
a_{21} & a_{22} & a_{23} & a_{24}\\
a_{31} & a_{32} & a_{33} & a_{34}\\
a_{41} & a_{42} & a_{43} & a_{44}
\end{bmatrix}=\begin{bmatrix}
x' & y' & z' & 1
\end{bmatrix}
\end{equation*}
Where $(x',y',z')$ is our answer. If we call the $4\times4$ matrix as $X$, then for shifting $x$ by $a$ co-ordinate, $y$ by $b$ and $z$ by $c$ co-ordinate,
\begin{equation*}
X=\begin{bmatrix}
1 & 0 & 0 & 0\\
0 & 1 & 0 & 0\\
0 & 0 & 1 & 0\\
a & b & c & 1
\end{bmatrix}
\end{equation*}
Instead of shifting, for scaling
\begin{equation*}
X=\begin{bmatrix}
a & 0 & 0 & 0\\
0 & b & 0 & 0\\
0 & 0 & c & 0\\
0 & 0 & 0 & 1
\end{bmatrix}
\end{equation*}
And finally, for rotating $\theta$ degrees around the $x$ axis following the right-hand rule (counter-clockwise direction)
\begin{equation*}
X=\begin{bmatrix}
1 & 0 & 0 & 0\\
0 & \cos\theta & -\sin\theta & 0\\
0 & \sin\theta & \cos\theta & 0\\
0 & 0 & 0 & 1
\end{bmatrix}
\end{equation*}
For $2$D rotation of $(x,y)$ by $\theta$ degree counterclockwise,
\begin{equation*}
\begin{bmatrix}
x & y
\end{bmatrix}\times\begin{bmatrix}
\cos\theta & \sin\theta \\
-\sin\theta & \cos\theta
\end{bmatrix}=\begin{bmatrix}
x' & y'
\end{bmatrix}
\end{equation*}
Where $(x',y')$ is our answer.
\subsection{Extended Euclidean Algorithm}
Returns the $\gcd$ of $a$ and $b$ with $ax+by=\gcd(a,b)$.
\lstinputlisting[language=C++]{src/extended_euclidean_algorithm.cpp}
\section{Useful Results}
\subsection{Finding directed path with fixed length}
Create the adjacency matrix and raise it's power to $k$, cell $(u,v)$ will give the number of distinct path with length $k$ connecting vertex $u$ and $v$ (direction from $u$ to $v$).

\end{document}
